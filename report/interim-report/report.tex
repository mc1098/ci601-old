\documentclass[12pt, a4paper]{article}

\usepackage[utf8]{inputenc}
\usepackage{bookmark}
\usepackage{nameref}

\hypersetup{
  colorlinks=true,
  linkcolor=black,
  citecolor=black,
  urlcolor=blue,
}

\usepackage[
  backend=bibtex,
  sortcites=true,
  sorting=none,
  abbreviate=false,
  isbn=false,
  url=false,
  style=authoryear,
  doi=false]{biblatex}
% raggedright to fix bibliography url hbox overflow
\appto{\bibsetup}{\raggedright}
\addbibresource{bibliography.bib}

\usepackage{graphicx}
\graphicspath { {images/} }

\title{A Rusty HTTP Server Library}
\author{Max Cripps}
\date{\today}


\begin{document}

\maketitle 

\section{Aims and Objectives}

In order to develop this project I will need to investigate several RFCs that define the standards
for an HTTP 1.1 server. As most HTTP servers are multithreaded, I will also need to investigate multithreaded
implementations that I can use when using Rust. 

The aim of this project is to develop a Rust library that can be used to create an RFC compliant
HTTP application server. The library will contain modular public interfaces so that users of the
library can build an HTTP server with different semantics then is provided. This library will also
have a high level interface for creating an HTTP server application that requires minimal code. I will
be creating multiple example projects to demonstrate using the library and to prove it works as intended,
see \emph{\nameref{sec:deliverables}} for more.

The library will not be \emph{"fully featured"} as providing this in the time frame would not be
possible. A library of this nature to be production ready would require more time and to be candid
more expertise in the area of security in order to make a viable solution that could rival what is
available in Rust.

\section{Requirements} \label{sec:requirements}

Most of the requirements are derived from RFCs that define the HTTP 1.1 protocol and RFCs in which
those depend on. The main RFCs that define the HTTP 1.1 protocol are as follows:
\begin{itemize}
  \item RFC7230 \emph{Message Syntax and Routing}
  \item RFC7231 \emph{Semantics and Content}
  \item RFC7232 \emph{Conditional Requests}
  \item RFC7233 \emph{Range Requests}
  \item RFC7234 \emph{Caching}
  \item RFC7235 \emph{Authentication}
\end{itemize}

All the RFCs listed above follow the requirement notation as defined in RFC2119 (\cite{rfc2119}),
this RFC defines keywords and how they should be interpreted in order to be RFC compliant.

The requirements are grouped under subsections using the MoSCoW prioritization (\cite{moscow}) method.

\subsection{Must Requirements}
\begin{itemize}
    \item\label{mreq:rfc-compliance} RFC compliance
    \item\label{mreq:memory-and-panic-safety} Memory and panic safety
    \item\label{mreq:public-interface-documentation} Public interface documentation
    \item\label{mreq:request-parsing} Request parsing
    \item\label{mreq:configurable-request-validation} Configurable request validation
    \item\label{mreq:status-code-based-errors} Status code based error handling
    \item\label{mreq:response-built-from-request} Response built from request
    \item\label{mreq:response-serialization} Response serialization
    \item\label{mreq:tcp-communication} TCP communication
    \item\label{mreq:example-echo-server} Echo server (example deliverable)
    \item\label{sreq:example-file-server} File server (example deliverable)
\end{itemize}

One of the major over arching requirement is to make sure that this implementation is memory and panic safe, which
means that, as the server is running that it is not leaking memory due to the library implementation
and cannot access invalid memory addresses. Panic safety is a further guarentee that the implementation
will uphold memory safety even when panics occur. Rust allows for more fine-grained control over
memory management which comes with dangers and benefits, see more in
\emph{\nameref{sec:choosen-approach}}.

I think a library requires good documentation in order to be usable by others, so this became a must
requirement for this project - it benefits me while implementing too as the design is modular, so I
will likely be using my own documentation.

RFC7230 (\cite{rfc7230}) contains the majority of the syntax for the request and response messages
in the HTTP protocol. The syntax notation used in this RFC is the Augmented Backus-Naur Form (ABNF)
notation of RFC5234 (\cite{rfc5234}). Much of the HTTP standard is based around the message syntax
defined in this RFC, therefore parsing and forming messages is a large part of the implementation.

The implementation of the transport layer even effects how requests need to be parsed; The URI
will accept port numbers as the normal transport layer is assumed to be TCP, however, if the
implementation is not TCP then parsing of URIs need to be more restrictive.

RFC7231 (\cite{rfc7231}) defines the all Status Codes that should be supported by both servers and
user-agents. The Status Codes is a three-digit number which explains the result of the request made
by the user-agent. Many of the other RFCs cross-reference this one when stating intended behaviour
of the server in regard to what status code it should return to a user-agent when a given event
occurs, such as when trying to pass a URI that is too long for the server then the server should
return a 414 (URI Too Long). The importance of status codes meant that I wanted to add them as a
major part of the error handling from beginning which is why it is part of the requirements list.

\subsection{Should Requirements}

\begin{itemize}
    \item\label{sreq:server-logging} Server logging
    \item\label{sreq:multithreaded-request-handling} Multithreaded request handling 
    \item\label{sreq:example-sleep-server} Sleep server (example deliverable)
    \item\label{sreq:request-routing} Request routing
    \item\label{mreq:example-list-server} List server (example deliverable)
\end{itemize}

The URI, adapted from RFC3986 (\cite{rfc3986}), defined in RFC7230 is a major part of defining 
the resource target which will help when implementing the request routing which will be a process 
provided by the library so that specific functions can be called to handle the request based on its
resource target and the method of the request.

\subsection{Could Requirements}

\begin{itemize}
    \item\label{creq:request-body-deserialization} Request body deserialization
    \item\label{creq:request-query-deserialization} Request query deserialization
    \item\label{creq:caching-service} Caching service
    \item\label{creq:transport-layer-agnostic} Transport layer agnostic
    \item\label{creq:asynchronous-request-handling} Asynchronous request handling
    \item\label{creq:conditional-requests} Conditional Requests
    \item\label{creq:range-requests} Range Requests
\end{itemize}

RFC7232 (\cite{rfc7232}) and RFC7233 (\cite{rfc7233}) define more semantics and capabilities to the
request syntax defined in RFC7230. These two RFCs and their semantics are a lower priority due to the
size and scope of this project, however, the contents of the RFCs were still researched as part of
the planning and requirements refinement as I want to create the project so that these requirements
could be added if more time to implement them before the deadline.

RFC7234 (\cite{rfc7234}) defines the standards for caching the response of requests. In this project,
caching is a lower priority as third party caches are plentiful and can work with RFC compliant
HTTP messages. The benefit to adding a caching service would be to provide more "out of the box"
and to improve the performance when handling multiple of the same requests.

\subsection{Won't Requirements}

\begin{itemize}
    \item\label{wreq:connect-method-support} CONNECT method support
    \item\label{wreq:http2.0-support} HTTP 2.0 support
    \item\label{wreq:authentication} Authentication service
    \item\label{wreq:proxy-or-gateway} Proxy OR Gateway
\end{itemize}

RFC7230 provides details regarding routing of proxy and gateway implementations that
won't be used in this project.

RFC7235 (\cite{rfc7235}) is the RFC defining authentication and other security considerations. As
a developer who is not an expert in application security I felt that Authentication and HTTPS
requirements were a low priority for this project, however, for a project to be useful in production
security must be one of the fundamental building blocks especially for an HTTP application. HTTPS and
authentication would also be more difficult to showcase clearly and correctly as HTTPS would require
certification and authentication would require some storage as part of the examples.

\section{Deliverables} \label{sec:deliverables}

\begin{itemize}
    \item Source code 
    \item Documentation
    \item GitHub repository (confirmation of CI)
    \item Server application examples \footnotemark[1]
      \begin{itemize}
          \item Echo server 
          \item Sleep server 
          \item File server
          \item List server
      \end{itemize}
\end{itemize}

\footnotetext[1]{The examples delivered will depend on the implementation as the examples listed
demonstrate different features of the library and so if they are not implemented then the example
will not be part of the deliverables, this also means more examples might be provided if more
requirements are implemented}

The main deliverable is the source code for the library, which will be provided in a compressed zipped
folder with the public documentation. The source code would also be available on GitHub as a private
repository, where access will be granted to my supervisor and second reader - the repository will
be made public after the deadline so that moderators would have free access to review, if required.
The GitHub repository is also set up for CI using
\href{https://docs.github.com/en/actions/learn-github-actions/understanding-github-actions}{\emph{GitHub Actions}}
which will run all the unit tests and document tests which should help aid moderators that the code
is working as intended.

\subsection{Examples}

The library itself will be quite a bit of code and would be hard to assert that it works, especially
if you aren't used to reading Rust code, therefore, I will also be including examples of application
servers built using the library code so that they can be run and hand tested.

Each example provided as a deliverable will come with a README that explains how to run the example
and the purpose it is trying to convey. They will also provide a sequence of commands that can be
used to test the example as a guide.

\subsubsection{Echo Server}

This example will provide a simple proof that the application server can receive a request and then
provide a response with the same body content. This will help demonstrate that the parsing, TCP
connection and serialization of the response is working correctly for simple GET requests.

\subsubsection{Sleep Server}

This example will provide two routes; the first which will always take a number of seconds to return
a request and another that will return a response almost immediately. The reason for this example is
to help demonstrate that the server can handle multiple requests at the same time without blocking
the other. The sleeping route will cause a thread blocking action which could be, but not limited to,
a thread sleep - thread in this context doesn't necessarily denote an OS thread as in an asynchronous
runner the thread would be a green thread.

\subsubsection{File Server}

The file server example will provide a static path for file lookup which can be requested using a
GET request. This will demonstrate using the library to query the request model in order to get
the path for the request file and returning that file as a response.

\subsubsection{List Server}

The list server example will hold an in memory global list which can be updated using the REST
architectural style (\cite{rest}).
This will demonstrate request routing specifically for different methods and somewhat simulates how
a server could be designed to interact with a database. The main purpose is not to address the problems
of shared mutability of global state in this example, but it could be given enough time to do so as
this would be a problem that would occur in a real server application.

\section{Choosen Approach} \label{sec:choosen-approach}

The general approach in this project is to use software and tools that promote sharing and
co-operation, this was not just for the development but also for planning and analysis of the
planning. I feel this is more akin to the normal procedure in the industry and allows for teams of
multiple people of different roles to work in an efficient and cohesive manner, also it was important
for directors to be able to quickly understand who was doing what at any given time.

\subsection{Language and Development Tools}

One of the first choices and ultimately one of the most impactful was, what programming lanugage
would be best suited to this project. I wanted to assure that the library was memory safe and has
good consistent performance. Memory safety is important because a server application is often a
long-running program, so any memory leaks will cause an \emph{out of memory} error. This requirement
for memory safety made me consider managed languages such as Java or Go, however, with managed memory
comes the garbage collector and this can add some hiccups in performance as found by Discord (\cite{discord}).

I first learnt to code with C++ so this was a consideration, but I had not used it since smart pointers
first got introduced, and I have heard many things have changed. I had just started trying out Rust
after seeing it become the most loved programming language on StackOverflow, and the mix of low level
control with high level concepts and a sprinkle of functional influence had me convinced that Rust
would be a great language to use for this project. Rust also has a concept of \emph{Ownership},
which was inspired in part from Cyclone (\cite{cyclone}), that enables Rust to make memory safety
guarentees without needing a garbage collector and the best part or this concept is that it is
completely in the control of the developer and not a runtime.

Rust and the tools it provides were another part of the decision, Rust has a dependency management
tool called Cargo, but it does more than this it provides a high level command line interface for 
build, testing, benchmarking and running Rust code with documentation.

Source control is a must for development and I have used and know Git very well and use it mostly
with the source control hosting site GitHub. In this project I choose to use Git as I only have
experience with this and Team Foundation Server (TFS), the latter being an unpleasent experience of
legacy code trapped in a legacy source control system. I have good experience using GitHub and
find GitHub Actions as an associated CI tool easy to set up and use for projects that don't require
multiple complex environments. 

\subsection{Planning with Jira}

Jira is an agile project management \emph{Software as a Service} (SaaS). I used Jira in order to
plan my project. I used Jira over Trello as I wanted to perform my implementation in
sprints and not use a Kanban-style. I have also used Jira in placement year, and I am aware that it 
is used by many large numbers of companies, therefore, learning more about Jira in the process of this 
project will build up another skill (\cite{jira}).

I researched the information provided by Atlassian regarding \emph{Agile project management},
this includes information on epics, stories, and estimation and how Jira helps. This information
helped me decide that I wanted to use Jira for project management and integrate this with GitHub.
I used the Scrum approach in Jira as I wanted to implement the work in sprints so that after each
sprint I could review the work and state of the project with a different perspective
(\cite{jira-agile-info}). 

\subsection{Implementation}

I considered many options for how and in what order I'd be implementing this project. I choose
to implement the request parsing first, this is by far the largest and arguably most important part
of the implementation in regard of RFC compliance; as defined in RFC7230 HTTP is an interface designed
through defining the syntax of communication. Then after implementing the request parsing I will then
work on the shared models that will be used by both the request and response so will include implementing
deserialization and serialization for requests and responses respectively. Then from this module of
work I can create the "core" of the library which includes setting up TCP conncections and providing
configurable contexts that can effect the parsing of requests. 

A possible other implementation route I considered was to use existing code as dependencies for the
request and response models allowing me to focus on the "core" of the project which would have afforded
me more time to create a more feature rich library and focus on performance tuning in a multithreaded
environment. This was tempting, however, I felt that a big part of the nature of this project is to
research and implement from an RFC which meant that request parsing was an important part of the
project to implement myself.

\section{Plan} \label{sec:plan}


\subsection{Review of other Rust HTTP Servers}

\subsubsection{Actix}

% Fig for images - use this as an example.
% \begin{figure}[h]
%     \centering
%     \includegraphics[width=0.25\textwidth]{calculator}
%     \caption{a nice plot}
%     \label{fig:mesh1}
% \end{figure}
    
\newpage
\printbibliography

\end{document}
